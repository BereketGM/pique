\documentclass{article}
\usepackage{amsmath}
\usepackage{enumerate}
\usepackage{graphicx}
\author{R.Y.~Neches \& E.G.~Wilbanks} 
\title{Analyzing ChIP-seq data with Pique}

\addtolength{\oddsidemargin}{-.875in}
\addtolength{\evensidemargin}{-.875in}
\addtolength{\textwidth}{1.75in}

\addtolength{\topmargin}{-.875in}
\addtolength{\textheight}{1.75in}


\begin{document}
\maketitle

\newcommand{\imsize}{0.45\columnwidth}
\newcommand{\threeup}{0.26\columnwidth}
\newcommand{\cotwo}{$\text{CO}_{2}$}
\newcommand{\htwo}{$\text{H}_2$}
\newcommand{\otwo}{$\text{O}_2$}
\newcommand{\water}{$\text{H}_2\text{O}$}
\newcommand{\htwos}{$\text{H}_2\text{S}$}

\begin{abstract}
\end{abstract}

It was found that most peak finders designed for ChIP-seq experiments
are designed for ChIP-seq in eukaryotes. To make cost-effective use of
current sequencing capacity, they must be cleverly optimized to work
with sparse-coverage data, and must take into account the effect of
chromatin structure on the variation in background coverage. For
ChIP-seq in bacteria and archaea, the differences in genome structure,
organization and size make the models used for identifying coverage
enrichment in eukaryotic poorly suited for use in bacteria and
archaea. 

For example, on this data, CSDeconv has a running time on the order of
days for a single ChIP-seq mapping.

% Why not? Examples, evidence, hypothesis. 

Fortunately, many of the statistical challenges for ChIP-seq in
eukaryotes are simply not present in experiments using bacterial and
archaeal models; this is due in part to higher genome coverage --
typically in inverse proportion to genome size -- and in part to the
absence of non-random coverage variation due to highly structured
chromatin.

\section{Experimental design}



\section{Analysis}

% what we did

The 40-bp reads were quality filtered and quality trimmed and aligned
to the {\em Halobacterium salinarum sp. NRC1} reverence genome using
bowtie. Reads mapped in the forward and reverse orientation were
separated, and used to calculate distinct coverage tracks.

% design paradigm

Because ChIP-seq in archaea yields coverage several orders of
magnitude larger than in eukaryotic systems, we designed a system to
take maximum advantage of this fact. Given the importance of manual
curation and settings optimization, Pique provides output suitable for
use in the Gaggle Genome Browser. This permits convenient interactive
curation of the peak list. 

For simplicity of design, we have employed several standard algorithms
and design principles from signal processing.

\begin{itemize}

\item Raw data is normalized with respect to the background.

\item Apply a mask to the ChIP track to remove regions with ambiguous
  read mapping. For example, it is impossible to map reads to unique
  loci in highly repetitive or palindromic regions, such as IS
  elements. As a result, the coverage is impossible to measure
  unambiguously, and the regions must be excluded from downstream
  analysis. 

\item The ``DC'' component is removed using linear detrending
  (scipy.detrend). This removes effects due to coverage variation
  features larger than about 100Kb.

\item High-$k$ noise in coverage is removed using a Wiener-Kolmogorov
  filter. The filter delay $\alpha$ is chosen to approximate to the
  expected footprint size of the targeted protein.

\item A coverage amplitude cutoff calculated from the detrended
  background track such that any given locus is equally likely to be
  above or below the cutoff. Enrichment features are defined with
  respect to this coverage level.

\item A sliding window moving average is used to identify regions
  whose coverage level deviates from the background. Peaks usually
  contain gaps in coverage that with widths on the order of the
  experimentally selected fragment size; the window width is chosen to
  correspond to this size.

% FIXME : Lizzy needs to verify that this is indeed the case

\end{itemize}

These steps yield simple rectangular envelopes around putative regions
of enrichment. To determine if these enriched regions correspond to
binding events, we apply a very simple statistical model :

\begin{itemize}

\item Coordinates of enriched regions in a peak are offset between
  strands, with the forward strand enriched upstream of the reverse
  strand. The first condition of the model is that the envelops must
  be overlapping rectangles; the end coordinate of the forward strand
  envelope must fall within the reverse strand envelope, and the
  start coordinate of the reverse strand envelope must fall within the
  forward strand envelope. 

\item Enrichment that are thought to represent binding events produce
  a characteristic shape envelope, which we model using a sum over set
  of Gaussians.

\end{itemize}

If the putative peak passes all of the above tests, this means that
the peak ``looks'' like a peak. To make sure that we are not finding
horsies by gazing at clouds, we also require that the integral of the
coverage in the raw data within the putative peak region exceeds the
integral of the coverage in the background by a margin set by the
user. (Other tests for statistical significance may also work, be more
shiny, et cetera. For example, Monte Carlo simulations of random
subsamples of the ChIP track and the background track until a
coalescent is found. )

\section{Theory}

The choice of filter implies some specific assumptions about the
nature of the coverage noise. The Wiener-Kolmogorov filter was the
first and simplest statistical signal filter, first published by
Norbert Wiener in 1949, and independently derived in discrete-time
form by Andrey Kolmogorov in 1941. The approach assumes the existence
of two inputs; a ``true'' signal, and a noise source. Both are assumed
to be stationary stochastic processes combined additively. 

\section{How many peaks are there?}



\end{document}