\documentclass{bioinfo}
%\usepackage{amsmath}
%\usepackage{enumerate}
\usepackage{graphicx}
%\author{R.Y.~Neches \& E.G.~Wilbanks} 
%\title{Analyzing ChIP-seq data with Pique}

%\addtolength{\oddsidemargin}{-.875in}
%\addtolength{\evensidemargin}{-.875in}
%\addtolength{\textwidth}{1.75in}
%\addtolength{\topmargin}{-.875in}
%\addtolength{\textheight}{1.75in}

\copyrightyear{2011}
\pubyear{2011}

% This writeup is intended to be submitted to Bioinformatics as an
% Applicaiton Note under either the Gene Expression or Sequence
% Analysis. These are the instructions to authors :
%
% Application Notes (up to 2 pages; this is approx. 1300 words or 1000
% words plus one figure) : Applications Notes are short descriptions
% of novel software or new algorithm implementations, databases and
% network services (web servers, and interfaces). Software or data
% must be freely available to non-commercial users. Availability and
% Implementation must be clearly stated in the article. Authors must
% also ensure that the software is available for a full TWO YEARS
% following publication. Web services must not require mandatory
% registration by the user. Additional Supplementary data can be
% published online-only by the journal. This supplementary material
% should be referred to in the abstract of the Application Note. If
% describing software, the software should run under nearly all
% conditions on a wide range of machines. Web servers should not be
% browser specific. Application Notes must not describe trivial
% utilities, nor involve significant investment of time for the user
% to install.
%
% Software : If the manuscript describes new software tools or the
% implementation of novel algorithms the software must be freely
% available to non-commercial users at the time of submission, and
% appropriate test data should be made available. Availability must be
% clearly stated in the article. Authors must also ensure that the
% software and test data is available for a full TWO YEARS following
% publication. The editors of Bioinformatics encourage authors to make
% their source code available and, if possible, to provide access
% through an open source license (see www.opensource.org for
% examples). Authors should make every effort to use URLs that will
% remain stable. At the minimum, authors must provide one of:
% webserver, source code or binary.
%
% http://www.oxfordjournals.org/our_journals/bioinformatics/

\begin{document}
\firstpage{1}

\title[In a fit of pique]{Analyzing high coverage microbial ChIP-seq data with Pique}
\author[Neches \textit{et~al}]{R.Y.~Neches\,$^{1,3}$\footnote{to whom
    correspondence should be addressed}, E.G.~Wilbanks\,$^{1,3}$ and
  M.T.~Facciotti\,$^{2,3}$
\address{$^{1}$Microbiology Graduate Group, University of California, Davis.\\
  $^{2}$Department of Biomedical Engineering, University of
  California, Davis.\\$^{3}$Genome Center BOO, University of California, Davis.}

\history{Received on XXXXX; revised on XXXXX; accepted on XXXXX}

\editor{Associate Editor: XXXXXXX}

\maketitle

%\newcommand{\imsize}{0.45\columnwidth}
%\newcommand{\threeup}{0.26\columnwidth}
%\newcommand{\cotwo}{$\text{CO}_{2}$}
%\newcommand{\htwo}{$\text{H}_2$}
%\newcommand{\otwo}{$\text{O}_2$}
%\newcommand{\water}{$\text{H}_2\text{O}$}
%\newcommand{\htwos}{$\text{H}_2\text{S}$}

\begin{abstract}
\section{Motivation:}

Most peak finders designed for ChIP-seq experiments
are designed for peak detection in eukaryotic datasets. To make cost-effective use of current sequencing capacity, they must be cleverly optimized to work
with sparse-coverage data, and must take into account the effect of
chromatin structure on the variation in background coverage. For
ChIP-seq in bacteria and archaea, the differences in genome structure,
organization and size make the models used for identifying coverage
enrichment in eukaryotic poorly suited for use in bacteria and
archaea. For example, on this data, CSDeconv has a running time on the
order of days for a single ChIP-seq mapping.

% Why not? Examples, evidence, hypothesis. 

Fortunately, many of the statistical challenges for ChIP-seq in
eukaryotes are simply not present in experiments using bacterial and
archaeal models; this is due in part to higher genome coverage --
typically in inverse proportion to genome size -- and in part to the
absence of non-random coverage variation due to highly structured
chromatin.

\section{Results:}
\section{Availability:}
\section{Contact:} \href{ryneches@ucdavis.edu}{ryneches@ucdavis.edu}

\end{abstract}

\section{Introduction}
ChIP-Seq, the selective chromatin immunoprecipitation of DNA binding proteins and their cognate DNA followed by DNA sequencing, 

\section{Approach}
The 40-bp reads were quality filtered and quality trimmed and aligned
to the {\em Halobacterium salinarum sp. NRC1} reverence genome using
bowtie. Reads mapped in the forward and reverse orientation were
separated, and used to calculate distinct coverage tracks.

% design paradigm

ChIP-seq in archaea yields coverage several orders of magnitude larger
than in eukaryotic systems, and so we designed a system to take
advantage of this fact. Given the importance of manual curation and
settings optimization, Pique provides output suitable for use in the
Gaggle Genome Browser. This permits convenient interactive curation of
the peak list. 

\begin{methods}
\section{Methods}

For simplicity of design, we have employed several standard algorithms
and design principles from signal processing.

\begin{itemize}

\item Raw data is normalized with respect to the background. The user
  selects one or more regions which are representative of the
  un-enriched background. Resequenced genomes often have coverage
  artifacts caused by features such as repetitive regions and gene
  dosage variation. For this reason, we advise the use of curated
  background regions. This operation is carried out by
  \texttt{piquify.py}.

\item A mask is applied to the ChIP track to remove regions with
  ambiguous read mapping. For example, it is impossible to map reads
  to unique loci in highly repetitive or palindromic regions, such as
  IS elements. As a result, the coverage is impossible to measure
  unambiguously, and the regions must be excluded from downstream
  analysis. (\texttt{pique.py})

\item The ``DC'' component is removed using linear detrending
  (scipy.detrend). This removes effects due to coverage variation
  features larger than about 100Kb. (\texttt{pique.py})

\item High-$k$ noise in coverage is removed using a Wiener-Kolmogorov
  filter. The filter delay $\alpha$ is chosen to approximate to the
  expected footprint size of the targeted protein. (\texttt{pique.py})

\item A coverage amplitude cutoff calculated from the detrended
  background track such that any given locus is equally likely to be
  above or below the cutoff. Enrichment features are defined with
  respect to this coverage level (\texttt{pique.py})

\item A sliding window moving average is used to identify regions
  whose coverage level deviates from the background. Peaks usually
  contain gaps in coverage that with widths on the order of the
  experimentally selected fragment size; the window width is chosen to
  correspond to this size. (\texttt{pique.py})


% FIXME : Lizzy needs to verify that this is indeed the case

\end{itemize}

These steps yield simple rectangular envelopes around putative regions
of enrichment. To determine if these enriched regions correspond to
binding events, we apply a very simple statistical model :

\begin{itemize}

\item Coordinates of enriched regions in a peak are offset between
  strands, with the forward strand enriched upstream of the reverse
  strand. The first condition of the model is that the envelops must
  be overlapping rectangles; the end coordinate of the forward strand
  envelope must fall within the reverse strand envelope, and the
  start coordinate of the reverse strand envelope must fall within the
  forward strand envelope. 

\item Enrichment that are thought to represent binding events produce
  a characteristic shape envelope, which we model using a sum over set
  of Gaussians.

\end{itemize}

\end{methods}

\begin{figure}[!tpb]%figure1
%\centerline{\includegraphics{fig01.eps}}
\caption{Caption, caption.}\label{fig:01}
\end{figure}

\section{Discussion}

If the putative peak passes all of the above tests, this means that
the peak ``looks'' like a peak. To make sure that we are not finding
horsies by gazing at clouds, we also require that the integral of the
coverage in the raw data within the putative peak region exceeds the
integral of the coverage in the background by a margin set by the
user. (Other tests for statistical significance may also work, be more
shiny, et cetera. For example, Monte Carlo simulations of random
subsamples of the ChIP track and the background track until a
coalescent is found. )


\section{Conclusion}
The choice of filter implies some specific assumptions about the
nature of the coverage noise. The Wiener-Kolmogorov filter was the
first and simplest statistical signal filter, first published by
Norbert Wiener in 1949, and independently derived in discrete-time
form by Andrey Kolmogorov in 1941. The approach assumes the existence
of two inputs; a ``true'' signal, and a noise source. Both are assumed
to be stationary stochastic processes combined additively. 

\section*{Acknowledgement}
\paragraph{Funding\textcolon} This project was funded by UC Davis startup funds to MTF, NSF graduate fellowship AWARD NUMBER to EGW and DARPA AWARD NUMBER TO RN.

\bibliographystyle{natbib}
\bibliographystyle{achemnat}
\bibliographystyle{plainnat}
\bibliographystyle{abbrv}
\bibliographystyle{bioinformatics}

\bibliographystyle{plain}

\bibliography{writeup}

\end{document}